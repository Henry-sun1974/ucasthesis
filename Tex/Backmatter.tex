%---------------------------------------------------------------------------%
%->> Backmatter
%---------------------------------------------------------------------------%
\chapter[致谢]{致\quad 谢}\chaptermark{致\quad 谢}% syntax: \chapter[目录]{标题}\chaptermark{页眉}
%\thispagestyle{noheaderstyle}% 如果需要移除当前页的页眉
%\pagestyle{noheaderstyle}% 如果需要移除整章的页眉

当敲下论文的最后一个字符,我环顾四周,看着镜子里憔悴的脸庞、鬓角新添的白发,反问自己,花费这么长的时间和那么大的精力攻读博士学位是为了什么?从我个人的学习经历来说,选择攻读博士学位,最初直白的想法,是为了提升自己的学历层次,更有利于职务晋升。当然,还有弥补自己人生的遗憾。

当年高考填报志愿之时,我带着满脑子“将军梦”进入了军校,然而陆军学院“四肢发达、头脑简单”的“米秒环”的四年痛苦经历,以及长达近十年的“重管理、轻科技”野战部队“小社会”生活,不仅塑造了我好胜的个性和“孤芳自赏”的心态,也暴露了自身高等教育功底不足的缺点,在我的身上留下莫大的遗憾。随着自己进入高层机关工作,与更高层级领导和社会各个阶层的接触面愈广,工作的视野圈日益开阔,这个缺点在自己愈加突出。尤其是先入山东大学、后进京工作后,提高整体素质的焦虑感和提升学术水平的紧迫感日益强烈。硕士毕业后,我曾想继续深造,但由于身份所限,再加上不负于领导对我的培养和赏识,出于补齐主官经历的需要,我来到梨城某集团军直属团任职。

在任职的几年里,职攻读博士学位的念头不时在勾起我心底的欲望。从当时的角度,功利地讲想为自己职务上的提升加分,从心里上还是为了弥补自己的遗憾。但同时还有一个深层次的原因,那就是将来伴随更大的机遇和更高平台,将自己的所学所创奉献给毕生挚爱的国家和军队。我能深切地感受到,这将是自己能够做到的最好方式!

2009年冬季开始,在原济南军区政治部主任王健中将的支持下,我与在北京航空航天大学读研的同学范志强博士军队政治工作信息化课题作研究。课题历经曲折却圆满完成,军报曾用整版的篇幅给予报道宣传,新华社、解放军报内参也曾上书高层。该课题在实际工作应用中达到了非常好的效果,可惜研发思想太超前且部署实施得太早,且正逢领导层换届。直至2020年多位部队领导询问源码和资料时,我才发现这项研究的系统光盘和数据盘因多次搬家而不知所踪。以至于每次与范博士谈及此事,身在航天信息区块链总监职位的范博士都摇头叹息不已,直笑我“生不逢时”。

回顾头来看,最大的收获,读博还是对我人生的反思和敬畏。冯友兰先生讲:人生所能有的成就有三:学问、事功、道德,即古人所谓立言、立功、立德。回想自己走过的路,拜学于三圣山水故里的泰城小镇中学,发伍于豫南周家山伍营,落于凤城之北,辗转泰莱九州通衢、行走于万寿路和玉泉山之间,在百年学府求学、申城五角场,本想在梨城“镀金”后续职上衔,却沉落于胶东五龙河畔,到蜗居于岛城一隅。立德,唯有达则兼济,贫则独善,它无正途。事功已成定式,只能概叹人生如戏,奈何命运不羁!而唯有学问,才是毕生之追求呀!。

读博的日子是快乐的,也是苦涩的。带给我最大的困难,不是学位课程的学习,而是课题研究带来的困难。首先是课题选择的困难。最初的课题是基于人工智能对特定人群的行为开展研究,在我投入大量经费、大展身手即将展开研究之际,我的职业生涯却面临了转变,这个课题也无法开展下去,只能另寻他径。其次是研究带来的困难。做科研如人生修行,既枯燥无味,又充满挑战,除了肯下笨功夫的人,没有捷径可走。初时查询论文的不便,课题题目的确定、创新点的选择,实验团队的组建,实验条件的搭建,实验标准的选择,对于在校博士生都不是小问题,对于我来说更是困难重重。多年的志不求易、事不避难的行事准则和数十年军旅生涯的磨砺,让我始终坚信:做困难的事终有所得!

读博的艰辛不是许多人所能想象的。这几年的时间里,有高光时刻,有失意之时,还有人事曲折。从年富力强,到白发爬上鬓角,从四十刚刚出头,到了快“知天命”的年龄,从主政一方到几乎从零开始,有时读书真的让我学习到怀疑人生。如果当初能预料到时间达八年之久的话,恐怕也不会有这样的选择。那熬坏了的灯泡、驼了的背和熬坏的胃,绝不是段子手所能涵括的,只有历经弥久之困且躬身驼行的人,才能有切肤之感受,才能体会到每一个字里行间、甚或每一个毛孔都和着汗水和心血。

感谢我的导师李学庆教授对我的关心、指导和帮助。李老师在我读硕士时就教授过我计算机专业课,自从我进入山大到今天博士毕业,每一点的进步都是与李老师的谆谆教诲分不开的。他渊博的学识、严谨的治学态度、温尔文雅的人格魅力以及对学术的执著追求,给每个接触到的人都留下深刻印象。撰写论文期间,从学位论文的选题、论文内容的选裁、论文结构的确定,到技术路线形成和论文的最终定稿,每一步都有李老师的辛勤付出。记得我首次将毕业论文寄给他,没想到次日凌晨2:05分收到了李老师修改意见。可以说,没有李老师的无私相助,也就没有本文最后的成果。

感谢我的父母。我的父亲是一个苦命人,不到一岁就失去了母亲,从完小毕业到自学成为一名大专生,最后成为了一名中学教师。多年徜徉在书海而乐此不疲的经历,在他的脑海中深深地刻下烙印,唯有读书才能改变命运。因此,父亲对我读博给予了超乎寻常的关心和关注。正是父亲这种刻苦自学的经历自小也潜移默化地影响着我,每次回到家中,他总是问长问短,生怕我不珍惜来之不易的读博机会。母亲从小生活在农村姊妹九人的大家庭,作为家里的老大,苦于姥爷重男轻女思想,母亲连一天学都没有机会上。但她对学习文化有着偏执的追求,对我们从小到大的教育倾注了无尽的精力。就是在那连饭都吃不饱的年代,她靠站在村里学堂的外面听课,学会了写自己的名字和简单的知识,直至现在虽近高龄,仍在不停地学习识字、练习绘画。尽管她不知道博士是何物,但她知道自己儿子所学所做必定是有益和正当的作为。每次见到我,都勉励我要下苦功夫、读好书、早点毕业。(先论是非,再论成败。)

在读博的几年时间里,我的妻子从省城到县城,再由县城孤身来到岛城,离开以前舒适的生活环境,离开教书育人的大学讲台,为了我的事业,独自一人带着幼子在岛城上学并照顾家中的老人,其中付出的辛勤和心血,让我有时连歉疚的话都说不出口。尤其在我面临最困难之时,她时终陪伴在我身边,和我一起渡过那段最坏的时光,鼓励我抛开外界是非、摒除心中的杂念,以超越的心态,利用好空闲期多读书、做好学问。

我和妻子研究生毕业于同一个学校,当时我在南校区,她在软件园。2013年在我攻读博士学位时,计算机学院搬到了软件学院校区,那时儿子读初一。2020年秋季,当我正在苦苦求索之时,儿子也幸运地到了山东大学软件学院学习。现今在我即将毕业之时,儿子也将赴加拿大多伦多大学攻读。从当时济南腊山脚下的小不点成为一米八多的男子汉,将近二十年的时间里,我们一家三口人都在同一个校区学习,也许这就是山大人的缘份吧。

我还要感谢网聪信息科技有限公司的创始人李惠民博士和北京银地信息科技有限公司的孙建东博士,在我学习期间,他们给予我提供了良好的科研条件,尤其是李惠民博士在研究过程中给予的指导。

读博期间,大家给予了我最大的包容、物质和精神上的帮助。济南军区副政委赵太忠中将、71146部队政委李景文少将、原青岛市食品药品监督管理局的鲍国春局长、张传增巡视员,71573部队的王大源、宋明辉、何顺秋、郭龙龙等在课题研究过程中给予了极大的支持和帮助,青岛市市场监管局网监处姜涛处长和全处人员在我撰写论文期间提供诸多便利并给予极大的支持,在此一并致谢!

读博的岁月里,行走的世界带给我的无尽的视野和不可多得的机遇,让我见到了想遇见的人,拥有了自己想要的东西,到了自己想去的地方,更成为自已梦想的人!

2022年1月30日,正是腊月廿八,次日便是大年初一。当我驾车走在滨海公路上,收音机里传出朴树的《只要平凡》,而我的心里尽是满足:在我悉心照料下身体安康的父母,优秀而又温良贤淑的妻子,学霸而又上进的儿子,当然还有碌碌无为而又自负的自己,虽快到知天命的年龄而一事无成,这何尝不是善良悲悯的母亲为我们经年的虔诚拜求和祖上积年的眷顾?!

这几年经历得太多,也体悟了不少。当一个人踮起脚尖靠近太阳的时候,全世界都挡不住他的阳光。当努劲全力追求精神的理想王国之时,这个世界都是你的同路人。对于一个有信仰的人,面对未来的路,无论是荆棘还是坦途,未来都将无所畏惧。

此时此刻,千言万语不如一句话:再次谢谢您们!

\chapter{作者简历及攻读学位期间发表的学术论文与研究成果}

\textbf{本科生无需此部分}。

\section*{作者简历:}

\subsection*{casthesis作者}

孙志东,河南郸城人,山东大学软件学院博士研究生。

\subsection*{ucasthesis作者}

李学庆,山东德州人,中国科学院力学研究所硕士研究生。

\section*{已发表(或正式接受)的学术论文:}

{
\setlist[enumerate]{}% restore default behavior
\begin{enumerate}[nosep]
    \item ucasthesis: A LaTeX Thesis Template for the University of Chinese Academy of Sciences, 2014.
\end{enumerate}
}

\section*{申请或已获得的专利:}

(无专利时此项不必列出)

\section*{参加的研究项目及获奖情况:}

可以随意添加新的条目或是结构。

\cleardoublepage[plain]% 让文档总是结束于偶数页,可根据需要设定页眉页脚样式,如 [noheaderstyle]
%---------------------------------------------------------------------------%
